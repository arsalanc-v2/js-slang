\subsection*{Stream Support}

The following stream processing functions are supported:

\begin{itemize}
\item \lstinline{stream_tail(x)}: \textit{Built-in}, assumes that the tail (second component) of the
  pair \lstinline{x} is a nullary function, and returns the result of
  applying that function.\\
\emph{Laziness:}  Yes: \lstinline{stream_tail} only forces the direct tail of a given
stream,
but not the rest of the stream, i.e. not the tail of the tail, etc.
\item \lstinline{stream(x1, x2,..., xn)}: \textit{Built-in}, returns a stream with $n$ elements. The
first element is \lstinline{x1}, the second \lstinline{x2}, etc.\\
\emph{Laziness:}  No: In this implementation, we generate first a
           complete list, and then a stream using \lstinline{list_to_stream}.
\item \lstinline{is_stream(x)}: Returns \lstinline{true} if
  \lstinline{x} is a stream as defined in the lectures, and
  \lstinline{false} otherwise.\\
\emph{Laziness:}  No: \lstinline{is_stream} needs to force the given stream.
\item \lstinline{list_to_stream(xs)}: transforms a given list to a stream.\\
\emph{Laziness:}  Yes: \lstinline{list_to_stream} goes down the list only when forced.
\item \lstinline{stream_to_list(s)}: transforms a given stream to a list.\\
\emph{Laziness:}  No: \lstinline{stream_to_list} needs to force the whole stream.
\item \lstinline{stream_length(s)}: Returns the length of the stream
  \lstinline{s}.\\
\emph{Laziness:}  No: The function needs to force the whole stream.
\item \lstinline{stream_map(f, s)}: Returns a stream that results from stream
  \lstinline{s} by element-wise application of \lstinline{f}.\\
\emph{Laziness:}  Yes: The argument stream is only explored as forced by
           the result stream.
\item \lstinline{build_stream(n, f)}: Makes a stream with \lstinline{n}
elements by applying the unary function \lstinline{f} to the numbers 0
to \lstinline{n - 1}.\\
\emph{Laziness:}  Yes: The result stream forces the applications of fun
            for the next element.
\item \lstinline{stream_for_each(f, s)}: Applies \lstinline{f} to every
  element of the stream \lstinline{s}, and then returns
  \lstinline{true}.\\
\emph{Laziness:}  No: \lstinline{stream_for_each} forces the exploration of the entire stream.
\item \lstinline{stream_reverse(s)}: Returns finite stream \lstinline{s} in reverse
  order. Does not terminate for infinite streams.\\
\emph{Laziness:}  No: \lstinline{stream_reverse} forces the exploration of the entire stream.
\item \lstinline{stream_append(xs, ys)}: Returns a stream that results from 
appending the stream \lstinline{ys} to the stream \lstinline{xs}.\\
\emph{Laziness:}  Yes: Forcing the result stream activates the actual append operation.
\item \lstinline{stream_member(x, s)}: Returns first postfix substream
whose head is equal to
\lstinline{x} (\lstinline{===}); returns \lstinline{null} if the
element does not occur in the stream.\\
 \emph{Laziness:}  Sort-of: \lstinline{stream_member} forces the stream only until the element is found.
\item \lstinline{stream_remove(x, s)}: Returns a stream that results from
given stream \lstinline{s} by removing the first item from \lstinline{s} that
is equal (\lstinline{===}) to \lstinline{x}.
Returns the original list
if there is no occurrence.\\
\emph{Laziness:}  Yes: Forcing the result stream leads to construction of each next element.
\item \lstinline{stream_remove_all(x, s)}: Returns a stream that results from
given stream \lstinline{s} by removing all items from \lstinline{s} that
are equal (\lstinline{===}) to \lstinline{x}.\\
\emph{Laziness:}  Yes: The result stream forces the construction of each next element.
\item \lstinline{stream_filter(pred, s)}: Returns a stream that contains
only those elements for which the one-argument function
\lstinline{pred}
returns \lstinline{true}.\\
\emph{Laziness:}  Yes: The result stream forces the construction of
            each next element. Of course, the construction
            of the next element needs to go down the stream
            until an element is found for which \lstinline{pred} holds.
\item \lstinline{enum_stream(start, end)}: Returns a stream that enumerates
numbers starting from \lstinline{start} using a step size of 1, until
the number exceeds (\lstinline{>}) \lstinline{end}.\\
 \emph{Laziness:}  Yes: Forcing the result stream leads to the construction of
            each next element.
\item \lstinline{integers_from(n)}: Constructs an infinite stream of integers
starting at a given number \lstinline{n}.\\
 \emph{Laziness:}  Yes: Forcing the result stream leads to the construction of 
            each next element.
\item \lstinline{eval_stream(s, n)}: Constructs the list
of the first \lstinline{n} elements
of a given stream \lstinline{s}.\\
\emph{Laziness:}  Sort-of: \lstinline{eval_stream} only forces the computation of
                the first \lstinline{n} elements, and leaves the rest of
                the stream untouched.
\item \lstinline{stream_ref(s, n)}: Returns the element
of stream \lstinline{s} at position \lstinline{n}, 
where the first element has index 0.\\
 \emph{Laziness:}  Sort-of: \lstinline{stream_ref} only forces the computation of
                the first \lstinline{n} elements, and leaves the rest of
                the stream untouched.
\end{itemize}
